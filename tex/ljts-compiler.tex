\documentclass[12pt,a4paper]{article}

\usepackage{lmodern}
\usepackage[utf8]{inputenc}

\usepackage[top=20mm,bottom=20mm,left=18mm,right=18mm]{geometry}

\usepackage{amsmath}
\usepackage{amsfonts}
\usepackage{amssymb}
\usepackage{array}
\usepackage[dvipsnames]{xcolor}
\usepackage{graphicx}
\usepackage{hyperref}
\usepackage{multicol}
\usepackage{parskip}

\definecolor{blue}{HTML}{000080}
\definecolor{turquoise}{HTML}{008080}

\begin{document}
\title{\bfseries Projet de Compilation}
\author{Lucas David, Sullivan Honnet, Théo Legras \& Jules Vittone}
\date{}
\maketitle

\part{Analyse lexicale}

Premièrement, nous pouvons définir des expressions régulières intermédiaires qui nous seront utiles pour éviter les expressions trop complexes:

\begin{tabular}{l}
    \verb!{digit}!   = \verb![0-9]!        \tabularnewline
    \verb!{letter}!  = \verb![A-Za-z0-9_]! \tabularnewline
    \verb!{newline}! = \verb![(\r\n?)|\n]! \tabularnewline
\end{tabular}

Dans un second temps, nous ne traitons pas le commentaire comme un token mais nous le reconnaissons pour pouvoir l'éliminer de l'analyse:

\begin{tabular}{l}
\verb!{commentary}! = \verb!\/{2}[^{newline}]*{newline}|\/\*([^\*]*|\*[^\/])*\*\/!
\end{tabular}


Et finalement, pour définir comment s'organise le projet de compilation, il faut réfléchir aux tokens que nous allons reconnaître. Aussi, pour chacun des tokens reconnu le nombre de caractère devra être compté pour garder un contexte d'erreur lors de la compilation. En voici, un exemple d'approche:

Soit $numLine$, $numCharOfLine$, $numChar$ des entiers initialisés à 0. \newline
Chaque fois que \verb!{newline}! est reconnu $numLine$ est incrémenté de 1. \newline
Chaque fois que \verb!{commentary}! est reconnu $numLine$ est incrémenté de 1 ou plus selon si le commentaire est multilignes ou non. \newline
Pour chaque token reconnut $numCharOfLine$ \& $numChar$ sont incrémenté du nombre de caractères du token (\verb|strlen(yytext)|). \newline
Et finalement pour tout autre caractère simple reconnu $numCharOfLine$ \& $numChar$ sont incrémentés de 1.

\newpage

Finalement, voici le tableau des différents tokens reconnus ainsi que leurs expressions régulières et types retournés respectifs (à noter que $\varnothing$ signifie que le token ne retourne pas de valeur):

\begin{center}
	\sffamily
	\begin{tabular}{|c|c|c|}
		\hline
		Token & Expression régulière & Type du token \tabularnewline
		\hline \hline
		\multicolumn{3}{|c|}{Mots clés réservés} \tabularnewline		
		\hline
		\textbf{CLASS}    & \verb|class|    & $\varnothing$ \tabularnewline
		\hline
		\textbf{DEF}      & \verb|def|      & $\varnothing$ \tabularnewline
		\hline
		\textbf{ELSE}     &  \verb|else|    & $\varnothing$ \tabularnewline
		\hline
		\textbf{EXTENDS}  & \verb|extends|  & $\varnothing$ \tabularnewline
		\hline
		\textbf{IS}       & \verb|is|       & $\varnothing$ \tabularnewline
		\hline
		\textbf{IF}       & \verb|if|       & $\varnothing$ \tabularnewline
		\hline
		\textbf{OBJECT}   & \verb|object|   & $\varnothing$ \tabularnewline
		\hline
		\textbf{OVERRIDE} & \verb|override| & $\varnothing$ \tabularnewline
		\hline
		\textbf{VAR}      & \verb|var|      & $\varnothing$ \tabularnewline
		\hline
		\textbf{RETURN}   & \verb|return|   & $\varnothing$ \tabularnewline
		\hline
		\textbf{THEN}     & \verb|then|     & $\varnothing$ \tabularnewline
		\hline \hline
		\multicolumn{3}{|c|}{Opérateurs arithmétiques} \tabularnewline
		\hline
		\textbf{PLUS}     & \verb|\+| & $\varnothing$ \tabularnewline
		\textbf{MINUS}    & \verb|-|  & $\varnothing$ \tabularnewline
		\textbf{MULTIPLY} & \verb|\*| & $\varnothing$ \tabularnewline
		\textbf{DIVIDE}   & \verb|\/| & $\varnothing$ \tabularnewline
		\hline \hline
		\multicolumn{3}{|c|}{Opérateurs relationnels} \tabularnewline
		\hline
		\textbf{RELATIONAL\_OPERATOR} & \verb!=|<>! & {\color{blue}enum} {\color{turquoise}operation}::{\color{turquoise}relational} \tabularnewline
		\hline \hline
		\multicolumn{3}{|c|}{Autres opérateurs} \tabularnewline
		\hline
		\textbf{ASSIGNMENT}     & \verb|:=| & $\varnothing$ \tabularnewline
		\textbf{MEMBER\_ACCESS} & \verb|\.|  & $\varnothing$ \tabularnewline
		\hline \hline
		\multicolumn{3}{|c|}{Idenifiants \& constantes} \tabularnewline
		\hline
		\textbf{IDENTIFIER} & \verb/{letter}({letter}|{digit})*/ & {\color{blue}class} {\color{turquoise}std}::{\color{turquoise}string} \tabularnewline
		\textbf{STRING}     & \verb|"[^"]*"|                     & {\color{blue}class} {\color{turquoise}std}::{\color{turquoise}string} \tabularnewline
		\textbf{CHAR}       & \verb|'[^']+'|             & {\color{blue}char}   \tabularnewline		
		\textbf{INTEGER}    & \verb|{digit}+|            & {\color{blue}int}    \tabularnewline
		\textbf{DOUBLE}     & \verb|{integer}\.{digit}*| & {\color{blue}double} \tabularnewline
		\hline
	\end{tabular}
\end{center}

\newpage

\part{Analyse syntaxique}

\section{Déclarations}

\subsection{Préambule}

\begin{multicols}{2}
	\sffamily
	\begin{tabular}{rcl}
		LOptParam & := & LParam                \tabularnewline
		    & \textbar & $\varepsilon$         \tabularnewline
		LParam    & := & Param \verb|,| LParam \tabularnewline
		    & \textbar & Param                 \tabularnewline
		Param     & := & \textbf{IDENTIFIER}   \tabularnewline
	\end{tabular}
	\begin{tabular}{rcl}
		LOptParamDecl & := & LParamDecl                    \tabularnewline
		        & \textbar & $\varepsilon$                 \tabularnewline
		LParamDecl    & := & ParamDecl \verb|,| LParamDecl \tabularnewline
		        & \textbar & ParamDecl                     \tabularnewline
		ParamDecl     & := & OptVar \textbf{IDENTIFIER} \verb|:| \textbf{IDENTIFIER} \tabularnewline
		OptVar        & := & \textbf{VAR}                  \tabularnewline
		        & \textbar & $\varepsilon$                 \tabularnewline
	\end{tabular}
\end{multicols}

\subsection{Déclaration d'une classe}

\begin{center}
	\sffamily
	\begin{tabular}{rcl}
		Class          & := & \textbf{CLASS} \textbf{IDENTIFIER} \verb|(| LOptParamDecl \verb|)| OptExtends \textbf{IS} \verb|{| ClassDef \verb|}| \tabularnewline
		OptExtends     & := & \textbf{EXTENDS} \textbf{IDENTIFIER}  \tabularnewline
		         & \textbar & $\varepsilon$                         \tabularnewline    
		ClassDef       & := & LOptField ClassConstructor LOptMethod \tabularnewline		
		ClassConstruct & := & \textbf{DEF} \textbf{IDENTIFIER} \verb|(| LOptParamDecl \verb|)| OptSuper \textbf{IS} \verb|{| Bloc \verb|}| \tabularnewline
        OptSuper       & := & \textbf{IDENTIFIER} \verb|(| LOptParam \verb|)|
	\end{tabular}
\end{center}

\subsection{Déclaration d'un objet}

{\sffamily
\begin{tabular}{rcl}
	Object          & := & \textbf{CLASS} \textbf{IDENTIFIER} \textbf{IS} \verb|{| ObjectDef \verb|}| \tabularnewline
	OptExtends      & := & \textbf{EXTENDS} \textbf{IDENTIFIER}   \tabularnewline
	          & \textbar & $\varepsilon$                          \tabularnewline
	ObjectDef       & := & LOptField ObjectConstructor LOptMethod \tabularnewline
    ObjectConstruct & := & \textbf{DEF} \textbf{IDENTIFIER} \textbf{IS} \verb|{| Bloc \verb|}|
\end{tabular}}

\subsection{Déclaration d'un champ}

{\sffamily
\begin{tabular}{rcl}
    LOptField & := & LField        \tabularnewline
        & \textbar & $\varepsilon$ \tabularnewline
    LField    & := & Field LField  \tabularnewline
        & \textbar & Field         \tabularnewline
    Field     & := & \textbf{VAR} \textbf{IDENTIFIER} \verb|:| \textbf{IDENTIFIER} \verb|;|
\end{tabular}}

\subsection{Déclaration d'une méthode}

{\sffamily
\begin{tabular}{rcl}
    LOptMethod  & := & LMethod        \tabularnewline
          & \textbar & $\varepsilon$  \tabularnewline
    LMethod     & := & Method LMethod \tabularnewline
          & \textbar & Method         \tabularnewline
    Method      & := & OptOverride \textbf{IDENTIFIER} \verb|(| LOptParamDecl \verb|)| MethodEnd \tabularnewline
    MethodEnd   & := & \verb|:| \textbf{IDENTIFIER} \textbf{ASSIGNMENT} Expr                     \tabularnewline
          & \textbar & OptReturn \textbf{IS} Bloc   \tabularnewline
    OptOverride & := & \textbf{OVERRIDE}            \tabularnewline
          & \textbar & $\varepsilon$                \tabularnewline
    OptReturn   & := & \verb|:| \textbf{IDENTIFIER} \tabularnewline
          & \textbar & $\varepsilon$
\end{tabular}}

\subsection{Expressions et instructions}

\subsubsection{Expressions}

On connait le principe pour faire apparaitre la priorité et l'associativité des opérateurs mais vu que Bison l'intègre "en dehors" de la définition des règles on va faire de même (pour consulter notre liste la priorité et l'associativité des opérateurs, autant directement consulter celle du C++: \url{https://en.cppreference.com/w/cpp/language/operator_precedence}).

{\sffamily
\begin{tabular}{rcl}
    Expr  & := & Expr \textbf{RELATIONAL\_OPERATOR} Expr                         \tabularnewline
    & \textbar & Expr \textbf{PLUS} Expr                                         \tabularnewline
    & \textbar & Expr \textbf{MINUS} Expr                                        \tabularnewline
    & \textbar & Expr \textbf{MULTIPLY} Expr                                     \tabularnewline
    & \textbar & Expr \textbf{DIVIDE} Expr                                       \tabularnewline
    & \textbar & \textbf{NEW} \textbf{IDENTIFIER} \verb|(| LOptParam \verb|)|    \tabularnewline
    & \textbar & \textbf{PLUS} Expr                                              \tabularnewline
    & \textbar & \textbf{MINUS} Expr                                             \tabularnewline
    & \textbar & \verb|(| \textbf{IDENTIFIER} Expr \verb|)|                      \tabularnewline
    & \textbar & \textbf{IDENTIFIER} \textbf{MEMBER\_ACCESS} \textbf{IDENTIFIER} \tabularnewline
    & \textbar & \textbf{IDENTIFIER} \textbf{MEMBER\_ACCESS} \textbf{IDENTIFIER} \verb|(| LOptParam \verb|)| \tabularnewline
    & \textbar & \verb|(| Expr \verb|)| \tabularnewline
    & \textbar & \textbf{IDENTIFIER}    \tabularnewline
    & \textbar & \textbf{INTEGER}       \tabularnewline
    & \textbar & \textbf{STRING}        \tabularnewline
\end{tabular}}

\subsubsection{Instructions}

    {\sffamily
    \begin{tabular}{rcl}
        Instr & := & Expr \verb|;| \tabularnewline
        & \textbar & Bloc          \tabularnewline
        & \textbar & \textbf{RETURN} \verb|;|      \tabularnewline
        & \textbar & Expr \textbf{ASSIGNMENT} Expr \verb|;|                   \tabularnewline
        & \textbar & \textbf{IF} Expr \textbf{THEN} Instr \textbf{ELSE} Instr \tabularnewline
    \end{tabular}}

{\sffamily
    \begin{tabular}{rcl}
        Bloc        & := & \verb|{| LOptInst \verb|}|                      \tabularnewline
              & \textbar & \verb|{| LOptVarDecl \textbf{IS} LInst \verb|}| \tabularnewline
        LOptVarDecl & := & LVarDecl         \tabularnewline
              & \textbar & $\varepsilon$    \tabularnewline
        LVarDecl    & := & VarDecl LVarDecl \tabularnewline
              & \textbar & VarDecl          \tabularnewline
        VarDecl     & := & \textbf{IDENTIFIER} \verb|:| \textbf{IDENTIFIER} Expr \verb|;| \tabularnewline
              & \textbar & \textbf{IDENTIFIER} \verb|:| \textbf{IDENTIFIER} \verb|;| \tabularnewline    
    \end{tabular}}

\end{document}